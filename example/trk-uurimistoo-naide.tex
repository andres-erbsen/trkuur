\documentclass{trkuur} % Reaalkooli vormistus. Muidu "report" või "article".

\usepackage{polyglossia} % Mitte-inglise keelte tugi
\setdefaultlanguage{estonian}
\usepackage{graphicx} % piltide lisamise jaoks

\usepackage[style=trkuur]{biblatex} % kasutatud kirjanduse genereerimine
\defbibheading{bibliography}{\chapter*{#1}} % lisame sisukorda ka
\addbibresource{tahtsad-viited-koikidele-toodele.bib} % viidete info fail

\title{Kuidas võimalikult produktiivselt \\ teha mitte midagi} % kahel real
\author{Hunt Kriimsilm}
\klass{11d}
\juhendaja{Nääritaat}


\begin{document}
\maketitle
\tableofcontents

\chapter*{Sissejuhatus}
Mina valisin selle teema kuna see huvitab mind väga ja on tänapäeval väga aktuaalne...

\chapter{Teo reetnud osa}
\emph{Esile toodud tekst on kaldkirjas.}
\textbf{Boldi saab ka teha, kuid on küsimus, milleks ometi.}
\texttt{See on väga selgelt loetav tekst.}
Vaadake veebilinki \url{http://xkcd.com}.

\chapter{Paranähtustest}
\section{Matemaatika}
Pythagoras ütles, et täisnurkses (joonisel \ref{taisk}) kolmnurgas kehtib seos $c^2 = a^2 + b^2$. Palju tõendeid selle kohta on tabelis \ref{pythtabel}.

\begin{figure}[h]
	\includegraphics[width=\textwidth]{pythagoras.png}
	\caption{Täisnurkne kolmnurk}
	\allikas{Mina ise ka ei mäleta, kust ma selle joonise tõmbasin}
	\label{taisk} % Selle järgi viidatakse, see rida peab olema pärast \caption
\end{figure}


\begin{table}[h]
	\caption{Eksperimentaalsed andmed}
	\label{pythtabel}
	\begin{tabular}{rr|r} % r - paremjoondatud tulp, | - püstjoon. lrcp
		Üks kaatet & Taine kaatet & Hüpotenuus  \\
		\hline
		3   &  4   &  5    \\
		12  &  5   &  13   \\
		6   &  8   &  10   \\
		7   &  24  &  25
	\end{tabular}
\end{table}

\section{kalajala}

\printbibliography

\appendix
\chapter{First appendix}
\cite{wiki}

\end{document}
